\documentclass{article}
\usepackage{ctex}
\usepackage{amsmath, amssymb}
\usepackage{geometry}

\title{魔方问题的证明思路}
\author{}
\date{}

\begin{document}
\maketitle

\section*{一.魔方基本定理}

\textbf{问题}:给定一个 4 元组 \( (v, r, w, s) \),其中 \( r, s \) 是8个角块和12个棱块各自的置换,\( v, w \) 是角块和对应棱块的方向,并且
\[
v \in C_3^8, \quad w \in C_2^{12},
\]
要确保它对应于魔方的一个打乱但可还原的状态,\( r, s, v, w \) 需要满足哪些条件?



\textbf{定理 1} (魔方基本定理)如上所述,一个 4 元组 \( (\vec{v}, r, \vec{w}, s) \)(\( r \in S_8 \), \( s \in S_{12} \), \( \vec{v} \in C_3^8 \), \( \vec{w} \in C_2^{12} \))对应于魔方上的某个可还原状态,当且仅当
\[
\begin{aligned}
(a) &\quad \operatorname{sgn}(r) = \operatorname{sgn}(s), \quad (\text{置换的奇偶性相等}) \\
(b) &\quad v_1 + \cdots + v_8 \equiv 0 \pmod{3}, \quad (\text{总扭转量守恒}) \\
(c) &\quad w_1 + \cdots + w_{12} \equiv 0 \pmod{2}, \quad (\text{总翻转量守恒})
\end{aligned}
\]

\section*{证明必要性}

为了简便起见,我们将 \( \vec{v} \) 记作 \( v \)。

首先我们证明必要条件。也就是说,我们假设 \( (v, r, w, s) \in S_V \times S_E \times C_3^8 \times C_2^{12} \) 表示魔方上一个(可还原)位置。接下来我们将证明其满足条件(a)到(c)。

令 \( g \in G \) 为一个将魔方从复原状态移动到由这个四元组表示的位置的元素。令 \( r = \rho(g) \) 且 \( s = \sigma(g) \)。我们知道,\( g \) 可以表示为基本移动 \( R, L, U, D, F, B \) 的乘积,记作 \( g = X_1 \dots X_k \),其中每个 \( X_i \) 等于 \( R, L, U, D, F, B \) 中的某一个。注意,如果 \( X \) 是任何一个这样的基本移动,那么 \( \operatorname{sgn}(\rho(X)) = \operatorname{sgn}(\sigma(X)) \)。由于 \( \operatorname{sgn}, \rho, \sigma \) 都是同态,因此我们有

\[
\operatorname{sgn}(r) = \operatorname{sgn}(\rho(g)) = \prod_{i=1}^k \operatorname{sgn}(\rho(X_i)) = \prod_{i=1}^k \operatorname{sgn}(\sigma(X_i)) = \operatorname{sgn}(\sigma(g)) = \operatorname{sgn}(s)
\]

这证明了(a)。

首先,容易验证(b)对基本移动\( R, L, U, D, F, B \) 成立。(详见$\mathbf{Lemma}\,\,\mathbf{ft}\_\mathbf{valid}$的形式化证明)注意到:

\begin{enumerate}
    \item 扭转守恒条件在(b)中对 \( (v_1, \dots, v_8) \) 成立,当且仅当对任意置换 \( P(p)(v) = (v_{(1)p}, \dots, v_{(8)p}) \) 成立。

    \item 如果 \( (v_1, \dots, v_8) \) 和 \( (v_1', \dots, v_8') \) 都满足(b)中的扭转守恒条件,那么它们的和也满足该条件。
\end{enumerate}

如上所述,将 \( g \) 表示为基本移动 \( R, L, U, D, F, B \) 的一个随机组合且每个操作出现的次数不限,记作 \( g = X_1 \dots X_k \),其中每个 \( X_i \) 等于 \( R, L, U, D, F, B \)。我们假设这个表达式是最简的,也就是说,我们选择 \( X_i \) 使得 \( k \) 尽可能小。这个 \( k \) 称为 \( g \) 的\textbf{长度}。(这个长度等同于从 \( g \) 到 \( G \) 的 Cayley 图中单位元的距离。)

现在我们根据长度通过数学归纳法证明(b)。对于长度 k=1 的所有随机组合,我们已经验证过该条件。

假设 \( k > 1 \)。通过给定的公式,用两个移动的方向来表示它们乘积的方向,我们有

\[
\vec{v}(X_1 \dots X_{k-1} X_k) = \rho(X_1 \dots X_{k-1})^{-1}(\vec{v}(X_k)) + \vec{v}(X_1 \dots X_{k-1})
\]

项 \( \rho(X_1 \dots X_{k-1})^{-1}(\vec{v}(X_k)) \) 通过上面的(i)项满足(b)中的扭转守恒条件。项 \( \vec{v}(X_1 \dots X_{k}) \) 通过归纳假设满足扭转守恒条件。它们的和也满足(b)中的扭转守恒条件(由上面的(ii)项)。这证明了(b)。

(c)的证明与(b)的证明非常相似,只是我们在这里使用了棱块的参数代替角块。(详见$\mathbf{Theorem}\,\,\mathbf{reachable}\_\mathbf{valid}$的形式化证明)
\section*{证明充分性}

现在,我们需要证明该定理的充分性。换句话说,假设(a)、(b)和(c)成立,我们需要证明存在一个对应的魔方操作组合以将其还原到初始位置。这部分的证明是主要依靠构造。

首先,我们证明一个特殊情况。假设 \( r \) 和 \( s \) 都是恒等置换,且 \( (w_1, \dots, w_{12}) = (0, \dots, 0) \)。存在一个仅改变两个角块方向而不改变位置的移动,并保持所有其他子块的方向和位置不变。例如,移动
\[
g = (R^{-1} D^2 R B^{-1} U^2 B)^2
\]顺时针旋转 120 度,使后下左角(bdl 角)顺时针旋转 240 度,并保持所有其他子块的方向和位置不变。此移动可以结合合适的对称性轻易地修改,以得到一个旋转任意一对角块的移动,同时保持所有其他子块的方向和位置不变。(所有移动的构造详见$\mathbf{Lemma1}\,\,\mathbf{001}\_\mathbf{UFL}$至\\$\mathbf{Lemma1}\,\,\mathbf{008}\_\mathbf{DBL}$的形式化证明)这些移动生成了所有满足(b)中扭转守恒条件的 8 元组。这证明了在 \( r \) 和 \( s \) 都为恒等置换且 \( (w_1, \dots, w_{12}) = (0, \dots, 0) \) 的情况下,该定理的充分性成立。

接下来,我们证明另一个特殊情况。假设 \( r \) 和 \( s \) 都是恒等置换,且 \( (v_1, \dots, v_8) = (0, \dots, 0) \)。

存在一个仅翻转两个棱块的移动,并保持所有其他子块的方向和位置不变。例如,移动
\[
g = L F R^{-1} F^{-1} L^{-1} U^2 R U R U^{-1} R^2 U^2 R
\]
翻转了前上棱块(uf 棱块)和上右棱块(ur 棱块),并保持所有其他子块的方向和位置不变。此移动可以结合合适的对称性轻易地修改,以得到一个翻转任意一对棱块,同时保持所有其他子块的方向和位置不变的移动。(所有移动的构造详见$\mathbf{Lemma2}\,\,\mathbf{001}\_\mathbf{UR}$至$\mathbf{Lemma1}\,\,\mathbf{012}\_\mathbf{UF}$\\的形式化证明)这些移动生成了所有满足(c)中翻转守恒条件的 12 元组。这证明了在 \( r \) 和 \( s \) 都为恒等置换且 \( (v_1, \dots, v_8) = (0, \dots, 0) \) 的情况下,定理的充分性成立。

作为这两个特殊情况的推论,当 \( r \) 和 \( s \) 都为恒等置换时,定理的充分性成立。

最后,我们证明最后一个特殊情况。假设 \( (v_1, \dots, v_8) = (0, \dots, 0) \) 且 \( (w_1, \dots, w_{12}) = (0, \dots, 0) \)。首先验证以下三个命题:

\begin{itemize}
    \item 给定任意三个棱块,有一个移动在这三个棱块上形成 3-循环,并保持所有其他子块的方向和位置不变。(详见$\mathbf{Lemma}\_\mathbf{31}$的形式化证明)
    \item 给定任意三个角块,有一个移动在这三个角块上形成 3-循环,并保持所有其他子块的方向和位置不变。(详见$\mathbf{Lemma}\_\mathbf{32}$的形式化证明)
    \item 给定任意一对棱块和任意一对角块,有一个移动在这两个棱块上形成 2-循环,在这两个角块上形成 2-循环,并保持所有其他子块的方向和位置不变。(详见$\mathbf{Lemma}\_\mathbf{16}$的形式化证明)
\end{itemize}



其次,根据抽象代数所学定理,我们知道 \( A_E \) 由上述的棱块 3-循环生成,\( A_V \) 由上述的角块 3-循环生成。换句话说,我们可以构造一个魔方位置,该位置与任意 4 元组 \( (r, s, 0, 0) \) 相关联,前提是 \( r \in A_V \) 且 \( s \in A_E \)。子群 \( A_E \times A_V \) 在 \( S_E \times S_V \) 中的指数为 4,因为 \( |S_n / A_n| = 2 \)。上述的第三种类型的移动,即棱-角 2-循环,不对应于魔方群中 \( A_E \times A_V \) 子集的任一元素,因为一个棱块的 2-循环是棱块的奇置换。因此,如果我们考虑由所有三类移动生成的 \( S_E \times S_V \) 的子群,我们将得到 \( S_E \times S_V \) 的全部,或者是一个严格包含 \( A_E \times A_V \) 且指数为 2 的真子群。第一种情况可以排除,因为它与条件(a)的奇偶性条件相矛盾。唯一严格包含 \( A_E \times A_V \) 的 \( S_E \times S_V \) 的指数为 2 的子群是满足条件(a)奇偶性条件的元素的子群。

因此,定理的充分性在 \( v \) 和 \( w \) 都为零的情况下是成立的。
\newpage 

最后,魔方基本定理是这些特殊情况的推论。(详见$\mathbf{Theorem}\,\,\mathbf{valid}\_\mathbf{reachable}$的形式化证明)

\textbf{推论}:无论魔方处于何种状态,总有一个移动不置换任何子块,但“还原”魔方的方向,使得 \( v \) 和 \( w \) 都为零。


于是,定理的证明完成。 \(\square\)

\section*{二.魔方可解问题}

\textbf{问题}:将魔方拆开,得到8个角块和12个棱块并将其随机拼成新的魔方,求新组装的魔方可复原的概率。\\

由魔方基本定理可知,新组装的魔方满足条件(a)的概率为$\frac{1}{2}$,满足条件(b)的概率为$\frac{1}{3}$,满足条件(c)的概率为$\frac{1}{2}$,且相互独立,故该魔法可复原的概率为$\frac{1}{12}$。
\end{document}